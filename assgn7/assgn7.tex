\documentclass{ctexart}
\usepackage{amsmath}
\def\dd{{\rm d}}
\begin{document}
\title{计算物理作业 7}
\author{刘畅, PB09203226}
\maketitle

{\bf [作业7]}: 用 Monte Carlo 方法计算如下定积分, 并讨论有效数字位数.
\[
\int^1_0 \sqrt{x+\sqrt{x}}\,\dd x
\]
\[
\int_{(x,y,z,u,v)\in[0,\frac{7}{10}]\times[0,\frac{4}{5}]\times[0,\frac{9}{10}]\times[0,1]\times[0,\frac{11}{10}]}
(6-x^2-y^2-z^2-u^2-v^2)
\]
\section{算法}
按照书上 Monte Carlo 积分的算法, 首先要在所要求的积分区域内生成均匀分布的随机点.
由于这一题中的两个积分都是在矩形区域上做的, 因此, 只要对每个积分变量产生(一维的)
均匀分布就达到要求了. 假设产生了 $N$ 个矩形上的均匀分布 $(\xi_{i_1},\ldots,\xi_{i_k})$,
Monte Carlo 积分的算法是
\[\label{monte}
\int f \approx \frac{1}{N}\left[\prod_{i=1}^{k}(b_i-a_i)\right] \sum_{i=1}^{N}
f(\xi_{i_1},\ldots,\xi_{i_k})
\]

\section{程序}
按照前面的算法, 首先要编写产生 $[0,a]$ 上均匀分布的例程: (\verb|main.c|)
\begin{verbatim}
/* uniform distribution over [0,a] */
double rand_uni(double a)
{
    return a * (double) rand() / (double) RAND_MAX;
}
\end{verbatim}

要用 Monte Carlo 方法计算 $\int^1_0\sqrt{x+\sqrt{x}}$,
按照算法 (\ref{monte}), 代码是非常直接的: (见 \verb|main.c|)
\begin{verbatim}
double integral_1(int nsteps)
{
    sum = 0.0;
    for (i = 0; i < nsteps; i++) {
        rand_nr = rand_uni(1.0);
        sum += sqrt(rand_nr + sqrt(rand_nr));
    }
    return sum / (double) nsteps;
}
\end{verbatim}
\verb|nsteps| 是式子 (\ref{monte}) 中的 $N$. 这个代码就是 (\ref{monte})
翻译成 C 语言, 所要没有什么好解释的.

第二个积分就是比第一个积分多用了4个变量, 其余的代码是完全一样的. 见
\verb|main.c:integral_2()|.
\begin{verbatim}
double integral_2(int nsteps)
{
    sum = 0.0;
    for (i = 0; i < nsteps; i++) {
        x = rand_uni(0.7);
        y = rand_uni(0.8);
        z = rand_uni(0.9);
        u = rand_uni(1.0);
        v = rand_uni(1.1);
        sum += 6.0 - x*x - y*y - z*z - u*u - v*v;
    }
    return sum * 0.7 * 0.8 * 0.9 * 1.1 / (double) nsteps;
}
\end{verbatim}

\section{结果}
在程序里我把 \verb|nsteps| 设成 \verb|1<<22|, 也就是 $2^{22} = 4194304$.
运行结果为, 对第一个积分
\[
\int f_1 \approx \input result_1
\]
第二个积分
\[
\int f_2 \approx \input result_2
\]

这些积分都是可以求出精确值的, 第一个积分用 Mathematica 积出的准确值为
\[
\input real_value_1 \approx \input numerical_value_1
\]
可以看到, 小数点后 4 位数和我们 Monte Carlo 积分的结果是一致的.

第二个积分的准确值为
\[
\input real_value_2 \approx \input numerical_value_2
\]
同样, 小数点后 4 位数和我们 Monte Carlo 积分的结果是一致的.

理论上可以证明, Monte Carlo 积分的有效数字正比于 $\frac{1}{\sqrt{N}}$. 这里
$\sqrt{N}=2^{11}=2048$, 因此我们预计小数点后4位数应该都是有效的. 这与前面
实验的结果一致.

\end{document}